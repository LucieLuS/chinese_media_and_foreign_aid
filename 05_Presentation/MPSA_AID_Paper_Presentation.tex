\documentclass{beamer}
\usepackage[utf8]{inputenc}
\usepackage[T1]{fontenc}
\usepackage{mathabx}
\usepackage{mathpazo}
\usepackage{eulervm}
\usepackage{natbib}



%% Load the markdown package
\usepackage[citations,footnotes,definitionLists,hashEnumerators,smartEllipses,tightLists=false,pipeTables,tableCaptions,hybrid]{markdown}
%%begin novalidate
\markdownSetup{rendererPrototypes={
 link = {\href{#2}{#1}},
 headingOne = {\section{#1}},
 headingTwo = {\subsection{#1}},
 headingThree = {\begin{frame}\frametitle{#1}},
 headingFour = {\begin{block}{#1}},
 horizontalRule = {\end{block}}
}}

%%end novalidate


\usetheme{Dresden}
\usefonttheme{serif}
\usecolortheme{rose}


\title{Friends Get More Money, Attention and Handshakes:
Chinese Foreign Aid, Xinhua and Diplomatic Visits}


\author{Lucie Lu, Miles D. Williams\\ University of Illinois at Urbana-Champaign}
\date {MPSA 2022 \\ April 8, 2022}


\begin{document}

\maketitle

\begin{frame}{Puzzle}

\begin{itemize}
    \item China seems to deliberately keep its foreign aid giving a secret.
    \item However, development finace arguably should promote "South-South cooperation" and improve China's image abroad.
    \item \textit{Such goals require transparency, so what should we make of its absence?}
\end{itemize}


    
\end{frame}

\begin{frame}{Research Question and Argument}

\begin{itemize}
\item \textbf{The Puzzle}: \textit{Absent formal modes of transparency, what evidence do we have that Chinese aid serves public diplomacy?}
\item \textbf{Argument}: \textit{We should observe complementarties between where China sends a greater number of diplomatic missions and that receive greater media coverage in outlets targeted at foreign audiences with where China gives greater foreign aid.}
\end{itemize}

\end{frame}


\begin{frame}{Argument Continued}

Two dimensions of Chinese public diplomacy efforts:

\begin{itemize}
    \item State-sponsored \textit{Xinhua} mediac coverage of developing countries intended for foreign audiences
    \item China’s bilateral South-South diplomatic activities

\end{itemize}


Developing countries that receive \textbf{higher} media coverage and diplomatic visit are strategically more important than the others. Hence, they get \textbf{more} aid.
    

\end{frame}

%%%%%%%%%%%%%%%%%%%%%%%%%%%%%%%%%%%%%%%%%%%%%%%%%%%%%%%%%%%%%%%%%%%%%%%%%%%%%%%%%%%%


\begin{frame}{Motivations behind Chinese Foreign Aid}


Mirror Western donors:

\begin{itemize}
    \item Countries in need \begin{footnotesize} (Dreher et al. 2018) \end{footnotesize}
    \item Countries import more goods from China \begin{footnotesize} (Dreher and Fuchs, 2015) \end{footnotesize}
    \item Countries with more natural resources, closer voting alignment with Beijing, more capacity to repay loans \begin{footnotesize} (Dreher et al., 2021) \end{footnotesize}
\end{itemize}

\end{frame}

\begin{frame}{Motivations behind Chinese Foreign Aid}

Like Western donors, China should have an interest in generating goodwill toward Beijing.

\begin{itemize}
    \item US cites benefits from public branding of tsunami relief to Indonesia in 2004.
    \item Surveys of policymakers show increasing influence of Beijing.
\end{itemize}

\end{frame}

\begin{frame}{Motivations behind Chinese Foreign Aid}

The obscurity seems to have more to do with capacity and logistics, rather than intention \begin{footnotesize} (Dreher et al. 2018) \end{footnotesize}. We
explore the \textbf{connections }between China’s bilateral development finance and other, visible,
activities linked to diplomatic and legitimacy-seeking objectives. 

\end{frame}

\begin{frame}{Media: First-level Agenda-setting}


\begin{itemize}
\item What's covered in the news signals to the public what is important \begin{footnotesize} (McCombs \& Shaw, 1972) \end{footnotesize}

\item Media coverage of the countries will lead to the countries as a whole becoming more
salient among the public. \begin{footnotesize} (Wanta et al., 2004) \end{footnotesize}

\item News editors as gatekeepers: the increased media salience suggests the salience of the foreign countries to the audiences and the \textbf{newsworthiness }to the news editors.
\end{itemize}

\end{frame}

\begin{frame}{Media: Second-level Agenda-setting}


\begin{itemize}
\item Media are critical in building states’ image to international audiences \begin{footnotesize} (Nye, 2008) \end{footnotesize}

\item Media coverage shapes public attitudes -- core objective of mediated public diplomacy. \begin{footnotesize} (Entman, 2008; Kiousis \& Wu, 2008) \end{footnotesize}

\item \textit{Xinhua} has dual roles: reporting news and building the China
image as part of Beijing’s “going out” strategy \begin{footnotesize} (Shambaugh, 2015) \end{footnotesize}

\item Its primary function is to transmit information. It reports on a variety of global issues
with different regional focuses to meet the standard of international news competitors -- \textbf{editors' choices} of what countries get more coverage
\end{itemize}

\end{frame}

\begin{frame}{Public Diplomacy}


\begin{itemize}
\item Visiting leaders can increase the awareness of themselves and their country among
citizens in the host country \begin{footnotesize} (Goldsmith \& Horiuchi, 2009; Goldsmith et al., 2021) \end{footnotesize}

\item Diplomatic visits usually come with policy agenda between two countries,
and the two are expected to cooperate in various domains through their negotiations.

\item Diplomatic visit is a signal of support and closeness of countries.

\end{itemize}

\end{frame}

%%%%%%%%%%%%%%%%%%%%%%%%%%%%%%%%%%%%%%%%%%%%%%%%%%%%%%%%%%%%%%%%%%%%%%%%%%%%%%%%%%%%


\begin{frame}{Hypotheses}

\begin{itemize}

\item Both media exposure and public diplomacy involve limited time and resources,
we expect that Beijing will choose on whom to cast the spotlight, and whom to visit,
strategically. 

\item We expect China to prioritize giving foreign
aid to its “friends” -- countries where China has developed strong relationships through
public diplomacy efforts -- and to countries it seeks to highlight to Western audiences.

\end{itemize}

\end{frame}


\begin{frame}{Hypotheses} 

\begin{itemize} 
    \item \textit{H1:} When developing countries receive more media exposure in \textit{Xinhua}, they are more
likely to receive more foreign aid from China.
    \item \textit{H2:} When developing countries host Chinese high-level diplomatic visits more
frequently, they are more likely to receive more foreign aid from China.
\end{itemize}

\end{frame}


%%%%%%%%%%%%%%%%%%%%%%%%%%%%%%%%%%%%%%%%%%%%%%%%%%%%%%%%%%%%%%%%%%%%%%%%%%%%%%%%%%%%


\begin{frame}{Data and Methods}

Unique recipient-year observations from 2001 to 2014

\begin{itemize} 

\item DV: bilateral aid data (AidData’s Chinese development finance dataset)

\item IV: number of diplomatic visits (AidData’s compiled yearly counts of bilateral official
diplomatic visits from China)

\item IV: counts of country media coverage (Millions of English edition \textit{Xinhua} news articles scraped by the Cline Center for Advanced Social Research)


\end{itemize}

\end{frame}


\begin{frame}{Data and Methods}

\begin{itemize} 

\item Covariates: aid recipient characteristics (GDP, presence of civil war, disasters, political and civil liberties) and bilateral characteristics between China and aid recipients (economic ties, alliances)

\item Model Specifications: $\text{ihs(aid}_{rt}) = \tau_t + \beta_1\text{ihs(coverage}_{rt}) + \beta_2\text{ihs(visits}_{rt}) + X_{rt}^\top \gamma + \epsilon_{rt}.$

\item Estimation: OLS (CR1 errors), Tobit (random recipient intercepts), Logit/Selection (random recipient intercepts), Level/OLS (CR1 errors), Lagged IV (CR1 errors), Lewbel IV (CR1 errors).

\end{itemize}

\end{frame}

%%%%%%%%%%%%%%%%%%%%%%%%%%%%%%%%%%%%%%%%%%%%%%%%%%%%%%%%%%%%%%%%%%%%%%%%%%%%%%%%%%%%

\begin{frame}{Results}

\begin{table}

\caption{Model Estimates}
\centering
\begin{tabular}[t]{lr>{}lr>{}lr>{}lr>{}lr>{}lr>{}l}
\toprule
\multicolumn{1}{c}{ } & \multicolumn{2}{c}{Linear} & \multicolumn{2}{c}{Tobit} & \multicolumn{2}{c}{Selection} & \multicolumn{2}{c}{Level} & \multicolumn{2}{c}{IV Lag} & \multicolumn{2}{c}{IV Lewbel} \\
\cmidrule(l{3pt}r{3pt}){2-3} \cmidrule(l{3pt}r{3pt}){4-5} \cmidrule(l{3pt}r{3pt}){6-7} \cmidrule(l{3pt}r{3pt}){8-9} \cmidrule(l{3pt}r{3pt}){10-11} \cmidrule(l{3pt}r{3pt}){12-13}
  & Est. & 95\% CI & Est. & 95\% CI & Est. & 95\% CI & Est. & 95\% CI & Est. & 95\% CI & Est. & 95\% CI\\
\midrule
Coverage & 0.63 & \includegraphics[width=0.67in, height=0.17in]{} & 1.29 & \includegraphics[width=0.67in, height=0.17in]{} & 0.18 & \includegraphics[width=0.67in, height=0.17in]{} & 0.11 & \includegraphics[width=0.67in, height=0.17in]{} & 0.67 & \includegraphics[width=0.67in, height=0.17in]{} & 0.51 & \includegraphics[width=0.67in, height=0.17in]{}\\
Visits & 0.93 & \includegraphics[width=0.67in, height=0.17in]{} & 1.54 & \includegraphics[width=0.67in, height=0.17in]{} & 0.21 & \includegraphics[width=0.67in, height=0.17in]{} & 0.36 & \includegraphics[width=0.67in, height=0.17in]{} & 1.08 & \includegraphics[width=0.67in, height=0.17in]{} & 1.38 & \includegraphics[width=0.67in, height=0.17in]{}\\
\bottomrule
\end{tabular}
\end{table}

\end{frame}

%%%%%%%%%%%%%%%%%%%%%%%%%%%%%%%%%%%%%%%%%%%%%%%%%%%%%%%%%%%%%%%%%%%%%%%%%%%%%%%%%%%%


\begin{frame}{Conclusion}


\begin{itemize}
    \item Despite a lack of formal transparency China's foreign aid serves a public role.
    \item Beijing's development finance complements its efforts at south-south diplomacy.
    \item And its development finance complements media coverage of aid recipients targted at Western audiences.
\end{itemize}

\textbf{Like Western donors, visibility foreign aid serves China’s interests.}
\end{frame}


\end{document}
